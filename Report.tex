\documentclass{article}
\usepackage{graphicx}
\usepackage{amsmath,amssymb}
\setlength{\parindent}{0pt}
\usepackage{listings}
\usepackage{lmodern}  % for bold teletype font
\usepackage{amsmath}  % for \hookrightarrow
\usepackage{xcolor}   % for \textcolor

\lstset{
  basicstyle=\ttfamily,
  columns=fullflexible,
  frame=single,
  breaklines=true,
}

\begin{document}

\title{Deep Reinforcement Learning Project 1 : Navigation}
\author{Rachel Schlossman}

\maketitle

\section{Learning Algorithm}
The learning algorithm is an implementation of a deep-Q network (DQN) as described in \cite{mnih2015human}. Following this paper, the action-values, $Q$, in iteration $i$ are updated using the following loss function:

\begin{equation}\label{eqn:loss}
L_i (\theta_i) = \mathbb{E}_{(s,a,r,s')~U(D)} \left[ \left( r + \gamma \max_{a'} Q(s',a';\theta_i^{-}) - Q(s,a;\theta_i) \right)^2 \right]
\end{equation}

where $\gamma$ is the discount factor . The minibatches of (state, action reward, next state) experiences, $(s, a, r, s')~U(D)$ are sampled uniformly from a replay buffer. The variable $\theta^{-}_i$ are the target network parameters. The target network parameters are set to the local Q-network parameters, $\theta_i$ every $N$ timesteps.

Eqn. \ref{eqn:loss} is implemented in the \textit{learn} function using Python and Pytorch as follows: 

\begin{lstlisting}[language=Python]
    def learn(self, experiences, gamma):
        """Update value parameters using given batch of experience tuples.

        Params
        ======
            experiences (Tuple[torch.Variable]): tuple of (s, a, r, s', done) tuples 
            gamma (float): discount factor
        """
        states, actions, rewards, next_states, dones = experiences

        ## TODO: compute and minimize the loss
        Q_target_next_states = self.qnetwork_target(next_states).detach().max(1)[0].unsqueeze(1)
        
        # target component of loss
        # Q_des = r + gamma * max_{a'} Q(s',a',w-)
        
        Q_des = rewards + (gamma * Q_target_next_states) * (1-dones)
        
        # Q_actual
        # Q(s,a,w)
        Q_act = self.qnetwork_local(states).gather(1,actions)

        # Compute loss
        # L = (r + gamma * max_{a'} Q(s',a',w-) - Q(s,a,w))^2
        loss = F.mse_loss(Q_des, Q_act)
        # Minimize loss
        self.optimizer.zero_grad()
        loss.backward()
        self.optimizer.step()
        
        # update target network
        self.soft_update(self.qnetwork_local, self.qnetwork_target, TAU)
\end{lstlisting}

\subsection{Hyperparameters}
The following heperparameters were used in the DQN implementation:

\begin{itemize}
\item  $\gamma$ = 0.995
]item $N$ = 4
\item learning rate = 5e-4
\item replay buffer size  = 1e5
\item minibatch size = 64
\item time constant for soft update of target parameters, $\tau$ = 1e-3
\end{itemize}


\section{Model Architecture}
The neural network employed has two hidden layers. The input layer (comprised of 37 nodes) passes through a linear transformation TODO

\section{Future Work}
The learning algorithm does not make use of Double Q-Learning, prioritized experience repoly or a dueling DQN architecture. Using one or more of these modifications could potentially improve the performance of the DQN agent. 

\bibliographystyle{plain}
\bibliography{bib}

\end{document}